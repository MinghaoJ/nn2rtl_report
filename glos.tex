%%% Acronyms %%%

\newacronym[%
    description={Convolutional Neural Net, a deep, feed-forward neural network implemented generally for image recognition; uses convolution operations.}%
]{cnn}{CNN}{Convolutional Neural Net}

\newacronym[%
    description={Directed Acyclic Graph, a graph with directed edges that also contains no cycles.}%
]{dag}{DAG}{Directed Acyclic Graph}

\newacronym[%
    description={Field Programmable Gate Array, an integrated circuit containing programmable logic cells.}%
]{fpga}{FPGA}{Field Programmable Gate Array}

\newacronym[%
    description={Intellectual Property, specifically referring to vendor-designed circuit blocks.}%
]{ip}{IP}{Intellectual Property}

\newacronym[%
    description={Logic Element, a unit of programmable logic cell implementing Boolean functions.}%
]{le}{LE}{Logic Element}

\newacronym[%
    description={Multiply-Accumulate, an operation performed by a convolution, computable as an inner product of two matrices.}%
]{mac}{MAC}{Multiply-Accumulate}

\newacronym[%
    description={Peripheral Component Interconnect Express, a high-speed data connection between system components such as CPU and memory and external components such as FPGAs and graphics cards.}%
]{pcie}{PCIe}{Peripheral Component Interconnect Express}

\newacronym[%
    description={Random Access Memory. Data storage that can be arbitrarily addressed efficiently.}%
]{ram}{RAM}{Random Access Memory}

\newacronym[%
    description={Block RAM. A type of memory block in FPGA fabrics that can be linked together to create memoriy banks of various dimensions.},%
    see={ram}%
]{bram}{BRAM}{Block RAM}

\newacronym[%
    description={Dynamic RAM. RAM that cannot retain data indefinitely and requires periodic ``refreshing,'' where the data contents are read and re-written (compare with SRAM).},%
    see={ram,sram}%
]{dram}{DRAM}{Dynamic RAM}

\newacronym[%
    description={Static RAM. RAM that retains its data so long as it is powered (compare with DRAM).},%
    see={ram,dram}%
]{sram}{SRAM}{Static RAM}

\newacronym[%
    description={Read Only Memory. Data storage that is programmed once on creation of the ROM.}%
]{rom}{ROM}{Read Only Memory}

\newacronym[%
    description={Rectified Linear Unit, a common activation function defined as $f(x) = \max(0, x)$.}%
]{relu}{ReLU}{Rectified Linear Unit}

\newacronym[%
    description={Register Transfer Level, a language which describes hardware elements at the register level.}%
]{rtl}{RTL}{Register Transfer Level}

\newacronym[%
    description={Robot Operating System, a framework for sensors, actuators, and processors to communicate.}%
]{ros}{ROS}{Robot Operating System}

\newacronym[%
    description={Very-Large-Scale Integration, the technology of integrating a very large number of transistors (today billions) on a single piece of semiconductor, typically silicon.}%
]{vlsi}{VLSI}{Very-Large-Scale Integration}

%%% Glossary entries %%%

\newglossaryentry{bias}
{
    name=bias,
    description={A neural network parameter that is added to the result of a kernel convolution.}
}

\newglossaryentry{conv_window}
{
    name=convolution window,
    description={The region in an input matrix being multiplied by the kernel.}
}

\newglossaryentry{filter}
{
    name=filter,
    description={Synonym for kernel.},
    see={kernel}
}

\newglossaryentry{kernel}
{
    name=kernel,
    description={a matrix of size n x n that is moved over the input image, taking the inner product with the input feature map to generate a point in the output feature map .}
}

\newglossaryentry{neural_net}
{
    name=neural network,
    description={Neural Network, a data processing paradigm that was inspired by the way the human brain processes information.}
}

\newglossaryentry{tensor}
{
    name=tensor,
    description={A generalized vector or matrix that could be of higher dimensions; represented as an n-dimensional array.}
}

\newglossaryentry{weight}
{
    name=weight,
    description={The value of an individual entry of a kernel.}
}
